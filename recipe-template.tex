\documentclass[fontsize=15pt,paper=a4]{scrdoc}

\usepackage[T1]{fontenc}
\usepackage[utf8]{inputenc} % UTF-8-Codierung
\usepackage[ngerman]{babel}
\usepackage{multicol}
\usepackage{enumitem}
\usepackage[left=1.5cm,
            top=1.5cm,
            right=1.5cm,
            bottom=1.5cm]{geometry}
\usepackage{xcolor}
\usepackage{FiraMono}
\renewcommand{\familydefault}{\ttdefault}
\addtokomafont{section}{\ttfamily\Large\centering}
\addtokomafont{subsection}{\ttfamily\large\centering}

\pagenumbering{gobble}

% Latex trennt dicktengleiche Schriften nicht, sodass über den Rand hinaus geschrieben wird. Das wird damit verhindert (aber was „sloppy“ genau macht, weiß ich auch nicht)
\sloppy

\begin{document}
\section*{\input{res/title.tex}}
\vspace{-15pt}
\begin{center}
    \scriptsize Datum: \today
\end{center}

\subsection*{Zutaten}
% TODO: Zwischenraum bei ungeraden Zutaten: Tabelle. Kann leicht mit & in res/ingredients.tex generiert werden. Bindestrich einfach per - <11-02-2024>
\begin{multicols}{2}
    \begin{itemize}[left=0pt, label={-}, itemsep=-0.15em]
        \input{res/ingredients.tex}
    \end{itemize}
\end{multicols}

\subsection*{Zubereitung}
\begin{enumerate}[left=0pt]
    \input{res/preparation.tex}
\end{enumerate}
\end{document}
